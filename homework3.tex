\documentclass[12pt]{article}
\renewcommand*{\familydefault}{\sfdefault}
\usepackage{listings}
\usepackage{amsmath}
\usepackage{fullpage}
\usepackage{tabularx}
\usepackage{graphicx}
\begin{document}
\title{CS540 Artificial Intelligence Homework 3}
\author{Yuchen Hou}
\maketitle

\section{Propositional logic}
\subsection{}
\begin{align*}
  r1: \neg likeRA \\
  r2: \neg likeRC \\
  r3: \neg likeDA \\
  r4: \neg likeDT \\
  r5: \neg likeBC \\
  r6: \neg likeDC
\end{align*}
\subsection{}
\begin{align*}
  \neg likeBC \land \neg likeRC \land \neg likeDC \implies likeJC \\
  r7: likeBC \lor likeRC \lor likeDC \lor likeJC
\end{align*}
\subsection{}
Add $\neg likeJC$ to KB and find empty clause:
\begin{align*}
  &r8: \neg likeJC \\
  &resolve(r7,r8): \\
  &r9: likeBC \lor likeRC \lor likeDC \\
  &resolve(r9,r5): \\
  &r10: likeRC \lor likeDC \\
  &resolve(r10,r6): \\
  &r11: likeRC \\
  &resolve(r11,r2): \\
  &r12: empty clause!
\end{align*}
so likeJC is true.
\subsection{}
Add to KB the previous clause and a clause similar to r7 and a implicit rule.
\begin{align*}
  &r13: likeJC \\
  &r14: likeBA \lor likeRA \lor likeDA \lor likeJA \\
  &r15: \neg likeJC \lor \neg likeJA
\end{align*}
Add $\neg likeBA$ to KB and find empty clause:
\begin{align*}
  &r16: \neg likeBA \\
  &resolve(r15, r14) \\
  &r17: likeRA \lor likeDA \lor likeJA \\
  &resolve(r16, r1) \\
  &r18: likeDA \lor likeJA \\
  &resolve(r18, r3) \\
  &r19: likeJA \\
  &resolve(r15, r13) \\
  &r20: \neg likeJA \\
  &resolve(r19, r20) \\
  &r21: empty clause!
\end{align*}
so likeBA is true.
\subsection{}
Add new knowledge to KB, using similar technique:
\begin{align*}
  &r22: likeBA \\
  &r23: likeBT \lor likeRT \lor likeDT \lor likeJT \\
  &r24: \neg likeBT \\
  &r25: \neg likeJT
\end{align*}
Add $\neg likeRT$ and find empty clause:
\begin{align*}
  &resolve(r25, r23) \\
  &r26: likeBT \lor likeRT \lor likeDT \\
  &resolve(r26, r24) \\
  &r27: likeRT \lor likeDT \\
  &r28: \neg likeRT \\
  &resolve(r27, r28) \\
  &r29: likeDT \\
  &resolve(r29, r4) \\
  &r30: empty clause!
\end{align*}
so likeRT is true.
\subsection{}
Add new knowledge to KB using similar technique:
\begin{align*}
  &r31 likeRT \\
  &r32 \neg likeRP (likeRT) \\
  &r33 \neg likeBP (likeBA) \\
  &r34 \neg likeJP (likeJC)\\
  &r35: likeBP \lor likeRP \lor likeDP \lor likeJP \\
\end{align*}
Add $\neg likeDP$ and find empty clause:
\begin{align*}
  &r36: \neg likeDP \\
  &resolve(r32, r36) \\
  &r37: likeBP \lor likeDP \lor likeJP \\
  &resolve(r33, r37) \\
  &r38: likeDP \lor likeJP \\
  &resolve(r34, r38) \\
  &r39: likeDP \\
  &resolve(r36, r39) \\
  &r39: empty clause! \\
\end{align*}
so likeDP is true.
\section{First order logic}
\subsection{}
\begin{align*}
  r1: \neg like(R,A) \\
  r2: \neg like(R,C) \\
  r3: \neg like(D,A) \\
  r4: \neg like(D,T) \\
  r5: \neg like(B,C) \\
  r6: \neg like(D,C) \\
\end{align*}
\subsection{}
p: person; g: game;
\begin{align*}
  \forall g \; \neg like(B,g) \land \neg like(R,g) \land \neg like(D,g) \implies like(J,g) \\
  r7: \forall g \; like(B,g) \lor like(R,g) \lor like(D,g) \lor like(J,g) \\
\end{align*}
\subsection{}
In the following sections, I will use a shorthand format of proof: \\
sentenceID:(resolvent1, resolvent2, \{substitution\})resultingClause. \\
Add $\neg like(J,C)$ and find empty clause.
\begin{align*}
  &r8: \neg like(J,C) \\
  &r9: (r7,\{g/C\}) like(B,C) \lor like(R,C) \lor like(D,C) \lor like(J,C) \\
  &r10: (r9,r2) like(B,C) \lor like(D,C) \lor like(J,C) \\
  &r11: (r10,r5) like(D,C) \lor like(J,C) \\
  &r12: (r11,r6) like(J,C) \\
  &r13: (r12,r8) empty clause! \\
\end{align*}
so like(J,C) is true.
\subsection{}
\begin{align*}
  \forall p,g1,g2 \; like(p,g1) \land g1 \neq g2 \implies \neg like(p,g2) \\
  r14: \forall p,g1,g2 \; \neg like(p,g1) \lor g1 = g2 \lor \neg like(p,g2) \\
\end{align*}
\subsection{}
Tell KB the following information:
\begin{align*}
  r15: A \neq C \\
  r16: A \neq P \\
  r17: A \neq T \\
  r18: C \neq A \\
  r19: C \neq P \\
  r20: C \neq T \\
  r21: P \neq A \\
  r22: P \neq C \\
  r23: P \neq T \\
  r24: T \neq A \\
  r25: T \neq C \\
  r26: T \neq P \\
  r27: like(J,C) \\
\end{align*}
Add $ \neg like(B,A)$ and find empty clause.
\begin{align*}
  &r28: \neg like(B,A) \\
  &r29: (r14,\{p/J, g1/C, g2/A\}) \neg like(J,C) \lor C = A \lor \neg like(J,A) \\
  &r30: (r29, r27, r18) \neg like(J,A) \\
  &r31: (r7, \{g/A\}, r1, r3, r30 r28) empty clause! \\
\end{align*}
So like(B,A) is true.
\begin{align*}
  r32: like(B,A) \\
\end{align*}
\subsection{}
Add $ \neg like(R,T) $ and find empty clause.
\begin{align*}
  &r33: \neg like(R,T) \\
  &r34: (r14,\{p/J, g1/C, g2/T\}, r27, r20) \neg like(J,T) \\
  &r35: (r14,\{p/B, g1/A, g2/T\}, r32, r17) \neg like(B,T) \\
  &r36: (r7,\{g/T\}, r35, r34, r33, r4) empty clause! \\
\end{align*}
So like(R,T) is true.
\begin{align*}
  r37: like(R,T) \\
\end{align*}
\subsection{}
Add $ \neg like(D,P)$ and find empty clause.
\begin{align*}
  &r38: \neg like(D,P) \\
  &r39: (r14,\{p/J, g1/C, g2/P\}, r27, r19) \neg like(J,P) \\
  &r40: (r14,\{p/B, g1/A, g2/P\}, r32, r16) \neg like(B,P) \\
  &r41: (r14,\{p/R, g1/T, g2/P\}, r37, r26) \neg like(R,P) \\
  &r42: (r7,\{g/P\}, r38, r39, r40, r41) empty clause! \\
\end{align*}
so like(D,P) is true.
\begin{align*}
  r43: like(D,P)
\end{align*}
\end{document}
